\documentclass[researchlab,palatino]{AIGpaper}
% Please read the README.md file for additional information on the parameters and overall usage of AIGpaper

%%%% Package Imports %%%%%%%%%%%%%%%%%%%%%%%%%%%%%%%%%%%%%%%%%%%%%%%%%%%%%%%%%%%%%%%%%
\usepackage{graphicx}					    % enhanced support for graphics
\usepackage{tabularx}				      	% more flexible tabular
\usepackage{amsfonts}					    % math fonts
\usepackage{amssymb}					    % math symbols
\usepackage{amsmath}                        % overall enhancements to math environment
\usepackage{amsthm}                         % Nutzung von Definition
\usepackage{hyperref}                       % Zeilenumbruch in URL
\usepackage{xurl}                           % Nach hyperref laden

%%%% optional packages
\usepackage{tikz}                           % creating graphs and other structures
\usetikzlibrary{arrows,positioning}
\tikzset{
    %Define standard arrow tip
    >=stealth',
    %Define style for argument
    args/.style={circle, minimum size=0.9cm,draw=black, thick,fill=white},
}


%%%% Author and Title Information %%%%%%%%%%%%%%%%%%%%%%%%%%%%%%%%%%%%%%%%%%%%%%%%%%%
\author{Robin Suxdorf \and Sebastian Bunge \and Johannes Krämer \and Emmanuelle Steenhof \and Alexander Kunze}

\title{Web Science - Die Qualität von Wikipedia-Artikeln}


%%%% Abstract %%%%%%%%%%%%%%%%%%%%%%%%%%%%%%%%%%%%%%%%%%%%%%%%%%%%%%%%%%%%%%%%%%%%%%

\germanabstract{
Erstmal nur ein Draft. Inhalte werden weiter abgestimmt. 
}

% use this if the document is written in english
%\englishabstract{}


\begin{document}

\maketitle % prints title and author information, as well as the abstract 


% ===================== Beginning of the actual text section =====================

\section{Einleitung}



Wikipedia-Artikel sind weit studierte und verwendete Artefakte, wenn es um die Textverarbeitung geht. Sie werden zum einen als Trainingsdaten für verschiedene Modelle des maschiniellen Lernens wie z.B. BERT \cite{BERTReference} und FastText \cite{DBLP:journals/corr/BojanowskiGJM16} verwendet. Zum Anderen wurden einige Modelle entwickelt, die sich mit verschiedensten Problemstellungen bzgl. Wikipediaartikeln beschäftigen, wie z.B. Hassan et al. \cite{shavarani2020multiclassmultilingualclassificationwikipedia} und Paramita et al. \cite{das2024languageagnosticmodelingwikipediaarticles}. .
%Bis hier habe ich geschrieben


%Im folgenden Text habe ich ein paar Dinge bearbeitet
Das Ziel dieses Projektpraktikums ist die Entwicklung eines Modells, um Wikipedia-artikel zu klassifizieren. Dabei sollen Artikel auf ihre Objektivität klassifiziert werden. Um das zu erreichen, wurden verschiedene Modelle des maschiniellen Lernens trainiert und evaluiert. Die Vorgehensweise zur Bearbeitung der Problemstellung entsprach dem Data Science Life Cycle. Abschnitt \ref{Aufgabenverteilung} und Abschnitt \ref{Organisation} gehen auf die organisatorischen Aspekte des Projektes ein. Abschnitt \ref{Datensatz} geht auf die Datenanalyse und die Definierung der Problemstellung ein. Darüber hinaus wird in diesem Abschnitt auch über weitere hinzugezogene Datensätze gesprochen. In Abschnitt \ref{Ansätze} werden die verwendeten Modelle vorgestellt. Die verwendeten Modelle waren der Bayes Klassifikator \ref{Bayes-Klassifikator}, die Support Vector Machine \ref{SVM}, die Logistische Regression \ref{Logistische Regression}, ein Convolutionales neuronales Netzwerk \ref{CNN} und ein Transformermodell \ref{Transformer}. Abschnitt \ref{Experimente} beschreibt die durchgeführen Experimente und in Abschnitt \ref{Ausblick} werden die Ergebnisse noch einmal zusammengefasst und bewertet.
%Bis hier habe ich bearbeitet.
 Darunter waren \\
Im Rahmen dieses Projekts bearbeiten wir folgende Teilaufgaben:

\begin{enumerate} 
\item \textbf{Analyse des Datensatzes und Identifizierung einer geeigneten Problemstellung}: Wir untersuchen den bereitgestellten Datensatz eingehend, um ein maschinelles Lernproblem zu formulieren, das mit den vorhandenen Daten gelöst werden kann. 
\item \textbf{Aufbereitung und Vorverarbeitung des Datensatzes}: Wir bereinigen und transformieren die Daten, um sie für die Modellierung vorzubereiten. 
\item \textbf{Anwendung von drei klassischen Methoden des maschinellen Lernens}: Basierend auf den Inhalten der Kapitel 2 und 3 des Kurses \glqq Einführung in Maschinelles Lernen\grqq{} implementieren wir drei klassische Algorithmen, um die identifizierte Problemstellung zu adressieren. 
\item \textbf{Anwendung eines Deep-Learning-Ansatzes}: Wir recherchieren einen geeigneten Deep-Learning-Ansatz, setzen diesen um und wenden diesen auf die Problemstellung an
\item \textbf{Entwickeln eines eigenen Ansatzes}: Im Rahmen dieser Ausarbeitung wird ein eigener Ansatz für die Problemstellung entwickelt und beschrieben. 
\item \textbf{Interpretation und Diskussion der Ergebnisse}: Basierend auf den bisherigen Resultaten entwickeln wir eine neue Idee für einen passenden Ansatz, beispielsweise eine neue Architektur für ein neuronales Netzwerk, die wir implementieren und anwenden. \end{enumerate}
%https://aclanthology.org/W09-3302.pdf

\section{Aufgabenverteilung}
\label{Aufgabenverteilung}
Im Rahmen des Praktikums haben alle Teammitglieder kontinuierlich an den verschiedenen Schritten des Projektes mitgeholfen. Dazu haben alle zu den Präsentationen und dem Abschlussbericht beigetragen. Im folgenden sind nicht genannte Abschnitte in Zusammenarbeit entstanden. Vortragsfolien wurden dem Arbeitsanteil entsprechend beigetragen. Die folgenden Abschnitte beschreiben, was jedes Gruppenmitglied individuell gemacht hat.

\subsection{Sebastian}
\subsection{Sebastian Bunge}
\label{sec:sebastian}

Sebastian Bunge implementierte eine modulare Projektstruktur, die sogenannte "Pipeline", inklusive einer modellübergreifenden Datenvorverarbeitung, Vektorisierung sowie Evaluationsmethoden und entwickelte den Bayes-Klassifikator. Er koordinierte die Anforderungen des Teams an die Pipeline und unterstützte bei der Integration weiterer Ansätze in diese. Er führte Stichprobenanalysen durch, um festzustellen, welche Vorverarbeitungsschritte und Vektorisierungsverfahren auf den klassischen machinellen Lernverfahren die besten Ergebnisse lieferten. Die Berichtsteile \ref{sec:sebastian}, \ref{sec:vorverarbeitung}, \ref{sec:bayes-klassifikator}, und \ref{sec:projektstruktur} dokumentieren seinen Anteil am Projektpraktikum.


\subsection{Johannes}
\subsubsection{Wikipedia-Dump}

Erstellung eines Konverters, um den Dump der englischsprachigen Wikipedia umzuwandeln in ein Format, dass mit dem Kaggle-Datensatz kompatibel ist.

\subsubsection{SVM}
Implementierung des SVM-Klassifikators.


\subsection{Alexander}
Alexander hat sich mit der ersten Datenanalyse der gegebenen Daten beschäftigt, um Anomalien oder andere Auffälligkeiten zu entdecken. Hierdurch wurde unter anderem die Herausforderung der Repräsentation der Sublabels entdeckt. Anschließend übernahm er die Entwicklung und Implementierung der logistischen Regression. Inklusive der Suche nach optimalen Parametern über Gridsearch. Abschließend hat Alexander weitere Trainingsdaten erstellt durch Easy Data Augmentation (EDA) und geprüft, ob diese beim Training der Modelle helfen würden. 

\subsection{Robin}
Robin war als Teamleiter tätig und übernahm die Leitung der Meetings sowie die Kommunikation mit den Praktikumsbetreuern {\ref{Organisation}}. Er unterstützte die Strukturierung des Projekts durch die Einführung einer abstrakten Klasse für Modelle, die als Grundlage für alle Ansätze diente {\ref{Vorverarbeitung}}. Zudem entwickelte er den Deep-Learning-Ansatz {\ref{KNN}} und erarbeitete den Zwischenvortrag, den er anschließend präsentierte.

\subsection{Emmanuelle}
Die organisatorische Unterstützung, die Emmanuelle vor allem leistete, war die Protokollführung bei den regelmässigen Treffen. Ausserdem teilte sie den Bericht in verschiedene Teildateien auf, sodass es beim parallelen Arbeiten nicht zu Problemen kam.
Beim Bericht ergänzte Emmanuelle den Abschnitt \ref{Transformer}. Sie ergänzte 3.1 Verwendete Programmiersprache, 3.2 Verwendete Tools und Techniken 4.2 Probleme des ursprünglicher Datensatz und 4.3 Weitere Daten. Sie schrieb auch beim 5 Abschnitt den einleitenden Text. Sie ergänze die Unterabschnitte von 7.1 Erfolglose Versuche.
Beim Zwischenvortrag entwickelte sie eine erste Version, die als Basis der Struktur des Vortrages dienen sollte. 
Emmanuelle implementierte den 5. Ansatz, in dem ein Transformer-Modell fine-getuned worden ist. Sie machte auch noch weitere Tests die für den 5. Ansatz gedient haben, aber nicht umbedingt im finalen Ergebnis zu sehen waren, weil sie nicht umbedingt erfolgreich waren.
Anfangs haben alle Analysen des Datensatzes durchgeführt, entsprechend auch Emmanuelle. Um einen geeigneten 5. Ansatz zu finden, machte sie weiterführende Analysen bzgl. des Datensatzes und weitere Recherchen, um Methoden zu finden, die auftretenden Probleme der ersten vier Ansätze zu beheben oder abzuschwächen. Diese Analysen führten auch zu entsprechenden Versuchen, in der Entwicklung von Ansätzen.
Ausserdem machte sie Versuche, um nach weiteren Datensätzen zu suchen. %Sollen wir die Datensätze noch ergänzen
Diese Versuche waren allerdings nicht besonders erfolgreich, weil das Ziel der Recherchen entweder nicht erfüllt wurde oder die gefundenen Artikel, die Probleme des Datensatzes nicht gelöst haben. 
%Sie half auch bei der Literaturrecherche für die Datenaugmentation.
Minimale Unterstützungen der anderen Ansätze, die sie erbrachte, waren das bereitstellen ihrer GPU für das Training des CNN und eine Rechercheunterstützung, um eine Erklärung für die sehr guten Ergebnisse des SVM-Ansatzes zu finden.

\section{Teaminterne Organisation}
\label{Organisation}
%Ich würde noch anmerken dass jedes Thema einen Leiter gekriegt hat.
Im Rahmen des Kick-Offs wurde beschlossen, dass Discord (bereitgestellt über Alexander Kunze) und Github (bereitgestellt von Robin Suxdorf) als Kollaborationsplattformen dienen. Ein wöchentlicher Jour-Fixe sichert den regelmäßigen Austausch. Jeder Teilnehmer verantwortet die Weiterentwicklung seiner Methode. Das bedeutet, er entwickelt die Methode weiter, gibt zum Jour-Fixe ein Update zum Stand und teilt mit, wenn es Herausforderungen gibt. Das Team unterstützt dabei jeden Leiter und gibt Feedback bei jeder Statusvorstellung.


\subsection{Verwendete Programmiersprache}
Damit ein guter Vergleich der Methoden stattfinden konnte, musste sich das Team auf eine Programmiersprache einigen. Dabei wurde Python gewählt. Die Wahl wurde getroffen, weil Python einige Programmbibliotheken hat, die sich für die im Projektpraktikum gestellten Aufgaben, gut eignen. Die Bibliotheken, die besonders zu dieser Entscheidung beigetragen haben, waren Scikit Learn \cite{skicitLearnRef}, welches besonders bei den klassischen Verfahren Anwendung fand und Pytorch \cite{pytorchRef}, welches hauptsächlich bei den Ansätzen mit Neuronalen Netzen Anwendung fand.

\subsection{Verwendete Tools und Techniken}
Damit die Zusammenarbeit funktionieren konnte, musste sich das Team auf eine einheitliche Vorgehensweise einigen. Darüber hinaus mussten einige Schritte durchgeführt werden, die notwendig waren, um die Verfahren des maschinellen Lernens durchführen zu können. Die Aufgabenstellung ist ein Teilgebiet des Natural Language Processing. Aus diesem Grund, mussten die Daten erst so aufbereitet werden, dass der Computer mit ihnen Berechnungen machen konnte. Die Methoden, die dafür Anwendung fand werden genauer in \textbf{Referenz ergänzen} erläutert. Weitere Vorverarbeitungsschritte waren die Bereinigung der Datensätze, um sie optimal vorzubereiten. Genauere Details werden in \textbf{Referenz ergänzen}angesprochen. Für die Erweiterung des Datensatzes wurde ein weiterer Datensatz hinzugezogen. Außerdem wurden die Daten augmentiert, worüber in \textbf{referenz ergänzen} mehr ergänzt wird. Anschließend wurden die maschinellen Verfahren angewandt, die in \textbf{Referenz ergänzen} genauer beschrieben werden. Danach wurden die Daten mit verschiedenen Metriken analysiert, die in \textbf{Referenz ergänzen} genauer erläutert werden.


\section{Datensatz und Problemstellung}
\label{Datensatz}

Bevor die Modellbildung starten konnte, mussten zunächst die Bedingungen geklärt werden. Das bedeutet, dass der Datensatz analysiert werden musste und eine entsprechende Problemstellung identifiziert werden musste. Im Folgenden wird zunächst der ursprüngliche Datensatz beschrieben und anschließend erklärt, wie dieser in Laufe des Projekts ausgebaut und ergänzt wird. Danach wird die daraus abgeleitete Problemstellung erläutert.

\subsection{Ursprünglicher Datensatz}
%Erter Satz geändert
Als Basis wurde ein Datensatz verwendet, welcher  verschiedene Wikipedia-Artikel enthält \cite{UrsprungDatensatz}. Der Datensatz wurde in zwei Dateien unterteilt. Die erste Datei enthielt \emph{promotional} (also werbend) Artikel und wurde noch weiter unterteilt. Dabei sind folgende Label vergeben:
\begin{itemize}
    \item advert – „Dieser Artikel enthält Inhalte, die wie eine Werbeanzeige verfasst sind.“
\item coi – „Ein Hauptautor dieses Artikels scheint eine enge Verbindung zu seinem Thema zu haben.“
\item fanpov – „Dieser Artikel ist möglicherweise aus der Sicht eines Fans geschrieben, statt aus einer neutralen Perspektive.“
\item pr – „Dieser Artikel liest sich wie eine Pressemitteilung oder ein Nachrichtenartikel oder basiert weitgehend auf routinemäßiger Berichterstattung oder Sensationslust.“
\item resume – „Dieser biografische Artikel ist wie ein Lebenslauf geschrieben.“
\end{itemize}
Der zweite Datensatz enthielt Artikel, die als \emph{good} klassifiziert worden sind. Die beiden Datensätze wurden zusammen verwendet, um die gesamte Datenbasis zu bilden. Insgesamt ergab das \textbf{Anzahl Datensätze einfügen} Daten. Die Label dieser Daten war wie folgt verteilt:


\textbf{2 Bilder einfügen: 1. Bild einzelne Label, 2. Bild Kombination der Label}
\\



\subsection{Probleme des ursprünglichen Datensatzes}
Wie man anhand des Diagramms \textbf{Auf Bild referenzieren} sehen kann, sind die Daten ungleich verteilt. Daten mit dem Label \textit{good} nehmen laut \textbf{Quelle einfügen} nur 0.59\% aller Wikipedia-Artikel ein. Allerdings sieht man anhand der Grafik, dass sie im Vergleich zu \textbf{Prozentsatz berechnen}. Zum einen führt das zu einem Verhältnis, das nicht der Realität entspricht, zum anderen kann das die Ergebnisse der trainierten Modelle verschlechtern.



\subsection{Weitere Daten}
Um ein gutes Modell zu erstellen, welches auf maschinellem Lernen basiert, braucht man entsprechende Datensätze. Wie in Abschnitt \textbf{Referenz einfügen} besprochen worden ist, ist der ursprüngliche Datensatz nicht ausreichend, um entsprechende Modelle zu trainieren. Aus diesem Grund wurden verschiedene Methoden ausprobiert und verwendet, um den ursprünglichen Datensatz zu erweitern.

\subsubsection{Datensatzerweiterung durch Wikipedia-Dump}
\label{WPDump}
Beim Wikipedia-Dump handelt es sich um einen von der Wikimedia-Foundation veröffentlichten Datensatz, der alle Wikipedia-Seiten umfasst. Durch Hinzuziehen des Dumps konnte die Datenbasis nicht nur auf einen aktuellen Stand gebracht werden, sondern wurde darüber hinaus auch um diejenigen Artikel erweitert, welche weder als \emph{good} noch als \emph{promotional} eingestuft sind.

Der Dump ist für die verschiedenen Wikipedia-Sprachversionen und in Varianten mit oder ohne Historie verfügbar. Hier wurde der Dump der englischsprachigen Wikipedia ohne Historie verwendet, der unter \url{https://dumps.wikimedia.org/enwiki/20241220/} zu finden ist. Er besteht aus zwei Dateien:

\begin{itemize}
    \item \emph{enwiki-20241020-pages-articles-multistream.xml.bz2} (circa 22 GB komprimiert und 97 GB entpackt): Enthält eine komprimierte XML-Datei mit allen 24.091.931 Seiten und den dazugehörigen Metadaten.
    \item \emph{enwiki-20241020-pages-articles-multistream-index.txt.bz2} (circa 250 MB komprimiert und 1 GB entpackt): Enthält eine Index-Datei mit 240.953 Offsets in der komprimierten XML-Datei zwischen denen jeweils 100 Seiten liegen.
\end{itemize}

Aufgrund seiner Größe konnte der Dump nicht komplett in den Arbeitsspeicher geladen werden. Unter Nutzung der Index-Datei wurden daher einzelne Abschnitte von jeweils 100 Seiten entpackt und verarbeitet.

Da die Seiten im Dump in Wiki-Sytax vorliegen, enthalten sie auch alle von den Wikipedia-Autoren eingesetzten Vorlagen (Templates). Jeder Artikel, der von der Wikipedia-Gemeinschaft als lesenswert (\emph{good}), exzellent (\emph{featured}) oder werbend (\emph{promotional}) klassifiziert wurde, enthält (mindestens) ein Template, anhand dessen diese Klassifizierung erkannt werden kann. Für \emph{good} und \emph{featured} ist dabei jeweils nur ein Template vorgesehen, dass in Wiki-Syntax als \textit{\{\{good article\}\}} beziehungsweise \textit{\{\{featured article\}\}} im Quellcode auftaucht, wobei abweichende Groß- und Kleinschreibung möglich ist. Für \emph{promotional} sind hingegen 21 verschiedene Templates möglich, die unterschiedliche Arten von werbendem Inhalt kennzeichnen. Außerdem lassen sich anhand von Templates und anderen Syntax-Elementen noch diejenigen Seiten identifizieren, welche keine Artikel darstellen, beispielsweise Begriffsklärungsseiten, Umleitungen, Kategorien und Benutzerseiten.

Es wurde ein Konverter entwickelt, der zunächst alle Seiten aus dem Dump verarbeitet, die zuvor genannten Nicht-Artikel-Seiten ausschließt und den Rest auf drei Kategorien verteilt: In die erste Kategorie \emph{good} fallen die wie zuvor beschrieben als lesenswert und exzellent gekennzeichneten Artikel; in die zweite Kategorie \emph{promo} die anhand der Templates als werbend erkannten und in die letzte Kategorie \emph{neutral} alle weiteren Artikel. Die zur Kategorisierung genutzten Templates sowie alle Zeilenumbrüche wurden aus dem Artikeltext entfernt, ansonsten wurde er unverändert übernommen.

Insgesamt ergab sich die folgende Aufteilung:
\begin{itemize}
    \item \emph{good}: 46.882
    \item \emph{promo}: 32.633
    \item \emph{neutral}: 6.611.303
    \item \emph{skipped}: 17.401.113 (keine Artikel)
\end{itemize}

Da die Klassen extrem ungleich verteilt sind (neutrale Artikel etwa um einen Faktor 200 häufiger als werbende), wurde anschließend noch eine zufällige Auswahl der Artikel jeder Klasse getroffen. Hierbei kam \textit{Reservoir Sampling} \cite{ReservoirSampling} zum Einsatz, bei dem die Elemente des Datensatzes einzeln gelesen werden, ohne dass deren Anzahl zuvor bekannt sein muss.

Um die Daten genau wie den Kaggle-Datensatz verarbeiten zu können, wurden sie in einem kompatiblen Format ebenfalls in CSV-Dateien geschrieben. Die Pipeline wurde so erweitert, dass sie neben der binären Klassifikation auch eine Mehrklassen-Klassifikation inklusive der Klasse \emph{neutral} unterstützt.

\subsubsection{Augmentierung der Daten}
Da es nicht sicher ist, ob der Wikimedia Dump das Problem vollkommen lösen kann, wurde neben dem Hinzuziehen weiterer Daten auch versucht die Daten zu augmentieren. Dabei sollten besonders untervetretene Klassen mehr Repräsentanten kriegen. Dabei wurden verschiedene Methoden ausprobiert, um die Daten zu augmentieren. Diese Methoden werden in \textbf{Referenz ergänzen} vorgestellt.

\subsection{Problemdefinition}
Das Ziel dieses Projekts ist die Entwicklung von Modellen zur automatisierten Klassifikation von Wikipedia-Artikeln als \emph{promotional} (werblich) oder \emph{nicht-promotional}. Dabei wird ebenfalls klassifiziert, wie ein Artikel promotional ist, also z.B. ob er eine Werbung, ein PR-Artikel usw. ist. Wikipedia strebt nach objektiven und neutralen Inhalten; daher ist die Identifizierung von Artikeln mit werbenden Charakter von großer Bedeutung, um die sachliche Qualität der Plattform zu gewährleisten.

\subsection{Zielsetzung}

Die Hauptziele des Projekts sind:

\begin{itemize} \item Entwicklung von drei klassischen maschinellen Lernmodellen und einem Deep-Learning-Modell zur Klassifikation von Wikipedia-Artikeln. \item Vergleich der Modelle anhand von Leistungsmetriken wie Genauigkeit, Präzision, Recall und F1-Score. \item Identifikation des Modells mit der besten Leistung für die gegebene Aufgabe. \end{itemize}



\section{Ansätze}
\label{Ansätze}
%https://shelf.io/blog/18-effective-nlp-algorithms-you-need-to-know/
Nachdem die passende Problemstellung festgestellt worden ist, war die nächste Aufgabe die passenden Ansätze zu auszuwählen. Dabei wurden im Rahmen des Projektpraktikum drei klassische Lernverfahren verwendet und ein Deep Learning Ansatz. Aufbauend auf diesen wurde ein 5. Ansatz konzipiert, der als Ziel hatte, die Schwächen der vorherigen Ansätze zu beheben. Im folgenden werden diese 5 Ansätze vorgestellt und anschliessend wird erklärt welche weiteren Techniken, besonders in Bezug auf Datenvorarbeitung und Datensatzergänzung verwendet worden sind.

\subsection{Bayes-Klassifikator}
\label{Bayes-Klassifikator}
\subsection{Bayes-Klassifikator}
\label{sec:bayes-klassifikator}

Der Bayes-Klassifikator ist ein probabilistisches Modell, das auf dem Bayes-Theorem basiert und häufig für Textklassifizierungsaufgaben verwendet wird \cite{maron_1961}. Trotz der Annahme, dass die Merkmale unabhängig voneinander sind - was in der Praxis nicht immer zutrifft - liefert der Klassifikator gute Ergebnisse.

Für die Experimente wurde hauptsächlich der Multinomial Naive Bayes-Klassifikator eingesetzt, da er sich besonders gut für die Verarbeitung von Textdaten eignet \cite{eyheramendy_2003}. Um optimale Parameter zu finden, kam GridSearchCV zur Hyperparameter-Optimierung zum Einsatz. Über eine Konfigurationsdatei können Hyperparameter wie Alpha und Fit-Prior gesetzt werden, wobei Grid Search ebenfalls über diese Konfiguration aktivierbar ist. Für Multilabel-Klassifikationen wurde der OneVsRestClassifier gewählt, um mehrere Labels gleichzeitig vorhersagen zu können - ergänzt durch einen optionalen RandomOverSampler, um ungleiche Labelverteilungen auszugleichen.

% SB: Überarbeitung des Abschnitts, Klarstellung von Begriffen wie GridSearchCV und OneVsRestClassifier


\subsection{Support Vector Machine}
\label{SVM}
\subsection{Support Vector Machine}
\label{SVM}

Die Support Vector Machine (SVM) ist ein überwachtes Lernverfahren zur Klassifikation, das darauf abzielt, eine die Klassen trennende Hyperebene mit maximalem Margin zu finden. Wie zum Beispiel in \cite{Joachims1998} beschrieben, eignet sich das Verfahren besonders gut zur Textklassifizierung.

Im Rahmen des Projekts wurden verschiedene Varianten der SVM von Scikit-Learn \cite{Pedregosa2011} getestet. Da der lineare Kernel auf den vorliegenden Daten leicht bessere Ergebnisse lieferte als die Alternativen RBF, Sigmoid und Polynomial, konnte die Implementierung \texttt{LinearSVC} eingesetzt werden. Diese unterstützt ausschließlich lineare Kernel, skaliert aber besser mit der Anzahl der Wikipedia-Artikel als die Implementierung \texttt{SVC}, welche kompatibel mit weiteren Kernel-Funktionen ist.

\texttt{LinearSVC} basiert auf der Bibliothek \texttt{LIBLINEAR}, die in \cite{Fan2008} beschrieben ist und löst das Optimierungsproblem
\begin{equation*}
  \min_{w,\, b} \frac{1}{2} w^T w + C \sum_{i=1}^{l} \left( \max(0, 1 - y_i (w^T x_i + b)) \right)^2.
\end{equation*}
Dabei ist \( w \in \mathbb{R}^d \) der Gewichtsvektor, \( b \in \mathbb{R} \) der Bias-Term, \( x_i \in \mathbb{R}^d \) ein Element aus dem Trainingsdatensatz mit Label \( y_i \in \{-1, 1\} \), \( l \) die Anzahl der Trainingsbeispiele und \( C > 0 \) der Regularisierungsparameter. Einem \( x \in \mathbb{R}^d \) wird dabei die Klasse \( sign(w^T x + b) \) zugewiesen. 



\subsection{Logistische Regression}
\label{Logistische Regression}
\subsection{Logistische Regression}

Für die Klassifikation von Wikipedia-Artikeln wurde eine logistische Regression als Modell eingesetzt. Bei der Unterscheidung der Hauptlabels reichte eine einfache Implementierung für die binäre Entscheidung. Bei der Klassifikation der Sublabels wurde der OneVsRest-Ansatz gewählt, um jeweils die binäre Entscheidung zu treffen. Um eine optimale Modellkonfiguration zu gewährleisten, wurde eine Hyperparameteroptimierung mittels GridSearch durchgeführt. Zur Feinabstimmung des Modells wurden folgende spezifische Parameter variiert. Für die Regularisierung wurden zwei Varianten getestet: Die \texttt{l1}-Regularisierung (Lasso) setzt irrelevante Merkmale auf null und führt somit zu einem sparsamen Modell. Die \texttt{l2}-Regularisierung (Ridge) reduziert die Gewichte irrelevanter Merkmale und verstärkt jene, die für die Klassifikation relevanter erscheinen. Neben der Regularisierung spielte das \textbf{Optimierungsverfahren} eine entscheidende Rolle. Zwei Solver wurden evaluiert: Der \texttt{liblinear}-Solver wurde als Standardmethode eingesetzt, da sie sowohl L1- als auch L2-Regularisierung unterstützt. Zusätzlich wurde \texttt{saga}, eine Erweiterung des Stochastic Average Gradient, getestet, die speziell für große Datensätze geeignet ist und ebenfalls mit beiden Regularisierungsarten kompatibel ist.
Ein weiterer wichtiger Parameter war die \textbf{maximale Anzahl an Iterationen} (\texttt{max\_iter}): \texttt{500} und \texttt{1000}.

Zur optimalen Abstimmung dieser Hyperparameter wurde eine \textbf{Grid Search} durchgeführt. Dabei wurden verschiedene Kombinationen von Regularisierung, Optimierungsverfahren und Iterationsanzahl systematisch getestet, um die beste Konfiguration für die logistische Regression zu bestimmen.



 



\subsection{Convolutional Neural Network}
\label{CNN}
\subsubsection{Datenvorverarbeitung}
Für die Datenvorverarbeitung der Wikipedia-Artikel wird zunächst ein Byte-Pair-Encoding-Algorithmus (BPE) - genauer der Tokenizer \texttt{cl100k\_base} aus der \texttt{tiktoken}-Bibliothek - angewendet. Dabei werden die Rohtexte in eine Sequenz von numerischen Token umgewandelt, die als Grundlage für die weitere Verarbeitung dienen. Anschließend wird der tokenisierte Text in einen Tensor konvertiert. Um eine einheitliche Eingabelänge zu gewährleisten, werden die Sequenzen auf eine fest definierte maximale Länge normiert: Kürzere Sequenzen werden mit einem speziellen Padding-Token aufgefüllt, während längere Sequenzen abgeschnitten werden. Die resultierenden Daten werden in einem PyTorch Dataset organisiert und mittels eines DataLoaders in Batches aufgeteilt, was eine effiziente Verarbeitung und einen reibungslosen Trainingsablauf ermöglicht.

\subsubsection{Modellarchitektur}
Das eingesetzte Modell basiert auf einem Convolutional Neural Network (CNN). Zunächst wird der tokenisierte Eingabetext über eine Einbettungsschicht (Embedding Layer) geleitet, die die diskreten Token in dichte, kontinuierliche Vektoren umwandelt. Auf den resultierenden Embeddings werden anschließend mehrere 1D-Convolutional Layers mit unterschiedlichen Filtergrößen angewendet. Diese Filter erfassen lokale Muster und n-Gramme im Text, wobei jede Faltungsoperation von einer ReLU-Aktivierungsfunktion gefolgt wird, um nichtlineare Zusammenhänge zu modellieren. Im Anschluss erfolgt ein Global Max-Pooling, das die wichtigsten Merkmale aus den erzeugten Feature-Maps extrahiert. Durch den Einsatz von Dropout wird zudem das Risiko eines Overfittings reduziert, bevor die gewonnenen Merkmale in einem Fully Connected Layer final zu Klassifikationslogits verarbeitet werden.

- Bild von Modellarchitektur? Oder von Faltungsoperation?

\subsubsection{Training}
Für das Training der Modelle kommen unterschiedliche Verlustfunktionen zum Einsatz, abhängig von der spezifischen Aufgabenstellung. Im binären Klassifikationsfall, bei dem entschieden wird, ob ein Artikel neutral oder nicht neutral ist, wird die \texttt{CrossEntropyLoss} verwendet. Im Multilabel-Fall, in dem nicht-neutrale Artikel in mehrere Kategorien wie \textit{fanpov} oder \textit{resume} eingeordnet werden, wird die \texttt{BCEWithLogitsLoss} genutzt, da ein Artikel gleichzeitig mehreren Klassen zugeordnet werden kann. Die Anpassung der Modellparameter erfolgt mithilfe des Adam-Optimierers. Aufgrund des hohen Rechenaufwands und der großen Datenmenge wird das Training auf einer GPU durchgeführt, um die Berechnungen erheblich zu beschleunigen.

\subsubsection{Hyperparameter-Optimierung}
Um die Performance des Modells weiter zu verbessern, wird eine systematische Hyperparameter-Optimierung durchgeführt. Dabei werden Parameter wie die Lernrate, die Anzahl der Filter, die Filtergrößen, die Dropoutrate sowie die maximale Sequenzlänge variiert und optimiert. Mithilfe von Cross-Validation und eines separaten Validierungsdatensatzes wird sichergestellt, dass die gewählten Hyperparameter zu einer guten Generalisierungsfähigkeit des Modells führen.
 
%Als Deep Learning Verfahren wurden Convolutional Neural Networks (CNN) verwendet. Zunächst war die Idee ein Neuronales Netz mit Rückkoppelung zu verwenden. Da das aber zu schlechten Ergebnissen führte wurden versuche mit einem Convolutional Neural Network gestartet. Diese hatten einen grösseren Erfolg, was dazu geführt hat, dass diese weiter als Ansatz verwendet worden sind.

\subsection{Fünfter Ansatz}
\label{Transformer}

Nachdem die ersten vier Ansätze gute Ergebnisse bei der binären Klassifikation gezeigt haben, zeigten sie Schwächen in der Multi-Label-Klassifikation. Nach weiteren Analysen, die das Ziel hatten, die genauen Unterschiede der einzelnen Promotional-Klassen zu erkennen, stellte sich raus, dass der Hauptunterschied weniger an der Struktur oder Ähnlichem liegt, sondern am Thema des Artikels zu liegen scheint. Nach weiteren Recherchen wurde der Bidirectional Encoder Representations from Transformers (BERT) als fünfter Ansatz gewählt, weil dieser, den beidseitigen Kontext miteinbezieht und daher besonders gut in der Erkennung von Kontext ist.
\paragraph{Transformers}
BERT ist ein Transformer. Transformer werden in einen Encoder und einen Decoder unterteilt. Encoder und Decoder bestehen jeweils aus $N$ Blöcken. Im Encoder sind diese in eine Multihead-Attention Schicht und ein Vorwärtsgerichtetes Netzwerk unterteilt. Die Schichten des Decoders haben ebenfalls diese 2 Unterschichten. Allerdings haben sie noch eine weitere Schicht, die eine Multihead-Attention über die Ausgabe des Encoders ausführt. Durch Attention kann der Transformer über seine Eingaben verwalten. Der Attention des Skalierten Skalarproduktes werden als Eingabe Abfragen $Q$, Schlüssel $K$ und Werte $V$ übergeben. Das Skalarprodukt berechnet sich dann wie folgt:
$${Attention(Q,K,V)} = {\frac{QK^T}{\sqrt{d_k}} V}$$
Die Attention wird mehrmals über Teilmengen der Eingabedaten berechnet und anschließend konkateniert. Dadurch entsteht Multi-head Attention. \cite{Vaswani2017}
(BERT) wurde so entwickelt, dass es auf einem riesigen nicht-gelabelten Datensatz trainiert wird. Das erlaubt dem Modell mithilfe einer weiteren Schicht auf eine spezifische Aufgabe abgestimmt zu werden. Diese Fähigkeit nennt man Transfer Learning. %BERT wurde zunächst mit unüberwachtem Lernen trainiert und kann anschließend gefine-tuned werden. %Eine der Eigenschaften, die dazu geführt haben, für dieses Praktikum BERT zu verwenden, war dass es auf u.a. auf dem Wikipedia Corpus trainiert worden ist. Dadurch kannte es den Aufbau der Artikel bereits und musste nur noch gefine-tuned werden, sodass es Daten klassifizieren kann. 
\cite{Devlin2018}.

\paragraph{Tokenizer}
%https://arxiv.org/pdf/1609.08144
%https://huggingface.co/learn/nlp-course/chapter6/6
Weil dieser Ansatz aufgrund des Transfer Learning gewählt worden ist, wurde statt der TF-IDF Vektorisierung der DistilBERT Tokenizer verwendet. Er basiert auf Word Piece. Word Piece unterteilt die Wörter und ergänzt spezialtokens. Daraus würde sich folgende Darstellung für das Wort token ergeben: \#t \#o \#k \#e \#n
%Daraufhin lernt der Tokenizer Mergeregeln.
Um die anschliessende Vereingung von Zeichen durchzuführen verwendet der Tokenizer die folgende Formel:
$$
    \text{score} = \frac{\text{freq\_of\_pair}}{\text{freq\_of\_first\_element} \cdot \text{freq\_of\_second\_element}}
$$
Wordpiece speichert die ermittelteten Wörter anschliessend, mit weiteren Spezialtokens als Vokabular ab.

\paragraph{Implementierung}
Das Verwendete Modell war DistilBERT \cite{Sanh2019}, welches den Vorteil hat, dass es die Genauigkeit von BERT weitgehend behält, aber eine deutlich schnellere Laufzeit hat. Obwohl das Modell mit verschiedenen Einstellungen getestet worden ist, wurden die Standardeinstellungen weitgehend behalten. Das Modell verwendet die Attention des skalierten Skalarproduktes. Es arbeitet mit einer maximalen Eingabelänge von 512 Tokens, beinhaltet 6 versteckte Schichten im Encoder von denen die Vorwärtsgerichteten Netze jeweils 3072 Neuronen beinhalten und verwendet als Aktivierungsfunktion die GELU-funktion \cite{Hendrycks2016}:
$$GELU(x) = x \cdot P(X \leq x) = x \cdot \Phi(x) $$
%Der verwendete Optimizer war AdamW und die Loss-funktion war der Cross-Entropy-loss:
%\begin{equation*}
%    \mathcal{L} = - \sum_{i=1}^{C} y_i \log(\hat{y}_i)
%\end{equation*}
 

%\subsection{Vortrainierte Transformer Modelle}
%Für die verwendete Methode wurden Vortrainierte Transformer Modelle verwendet. Dabei wurden verschiedene Transformer betrachtet. 

%\subsection{Datenaugmentierung}

\subsection{Transfer Learning und Vortrainierte Embeddings}

https://arxiv.org/pdf/1301.3781 Falls wir Word2Vec verwenden.

https://nlp.stanford.edu/pubs/glove.pdf falls wir GloVe verwenden.


https://arxiv.org/pdf/1607.04606 falls wir Fast Text verwenden.

https://arxiv.org/pdf/1802.05365 Falls wir Elmo verwenden wollen. Ich finde das sehr interessant.


https://arxiv.org/pdf/1810.04805 Falls wir Bert verwenden wollen. Berts Vorteile liegen auch stark darin dass es genau auf dem Wikipedia Corpus trainiert worden ist.

 
%https://arxiv.org/pdf/2308.00939

\section{Experimente}

Um die Datensätze miteinander vergleichen zu können, kamen verschiedene Metriken zu Anwendung


\subsection{Evaluationsmetriken}
\begin{enumerate}
    \item 
Sei $D = \{(x^{(1)}, y^{(1)}), \dots, (x^{(m)}, y^{(m)})\}$ ein Datensatz und $clf: \mathbb{R}^n \to \{0, 1\}$ ein (binärer) Klassifikator. Das \textbf{Genauigkeitsmaß} $acc$ von $clf$ bezüglich $D$ ist definiert durch
\begin{equation}
    acc(D, clf) = \frac{1}{m} \sum_{i=1}^{m} \left(1 - \left|y^{(i)} - clf(x^{(i)})\right|\right)
\end{equation}
    
\item 
Wir definieren

\begin{equation}
    TP(D, clf) = |\{i \mid y^{(i)} = 1, clf(x^{(i)}) = 1\}|
\end{equation}
\begin{equation}
    TN(D, clf) = |\{i \mid y^{(i)} = 0, clf(x^{(i)}) = 0\}|
\end{equation}
\begin{equation}
    FP(D, clf) = |\{i \mid y^{(i)} = 0, clf(x^{(i)}) = 1\}|
\end{equation}
\begin{equation}
    FN(D, clf) = |\{i \mid y^{(i)} = 1, clf(x^{(i)}) = 0\}|
\end{equation}

Die \textit{Konfusionsmatrix} von $clf$ bzgl. $D$ stellt die vier oben genannten Werte tabellarisch wie folgt dar:

\[
\begin{array}{|c|c|c|}
\hline
 & y = 1 & y = 0 \\
\hline
clf = 1 & TP(D, clf) & FP(D, clf) \\
clf = 0 & FN(D, clf) & TN(D, clf) \\
\hline
\end{array}
\]
\item
Sei $D = \{(x^{(1)}, y^{(1)}), \dots, (x^{(m)}, y^{(m)})\}$ ein Datensatz und $clf: \mathbb{R}^n \to \{0, 1\}$ ein ~(binärer) Klassifikator. Definiere
\begin{itemize}
    \item \textbf{Präzision:}
    \begin{equation}
    \text{prec}(D, clf) = \frac{TP(D, clf)}{TP(D, clf) + FP(D, clf)}
     \end{equation}
    \item \textbf{Recall:}
 \begin{equation}
    \text{rec}(D, clf) = \frac{TP(D, clf)}{TP(D, clf) + FN(D, clf)}
    \end{equation}
    \item \textbf{F1:}
    \begin{equation}
    \text{F1}(D, clf) = \frac{2 \cdot \text{prec}(D, clf) \cdot \text{rec}(D, clf)}{\text{prec}(D, clf) + \text{rec}(D, clf)}
    \end{equation}
\end{itemize}
\end{enumerate}
\label{Experimente}

\section{Ausblick}
\label{Ausblick}
Nachdem alle Experimente durchgeführt und evaluiert worden sind, werden die gesammelten Erfahrungen festgehalten.

\subsection{Erfolglose Versuche}
Um gute Modelle zu finden, die die Aufgabe lösen konnten, gab es einige Versuche, die zwar vielversprechend schienen, aber in der Praxis keine erfolgreichen Ergebnisse lieferten.

\subsubsection{Easy Data Augmentation}
\label{EDA}
Um das Ungleichgewicht bei unterrepräsentierten Sublabeln auszugleichen, wurde versucht, zusätzliche Texte mithilfe von Easy Data Augmentation \cite{Wei2019} zu generieren. Dabei wurden Synonymaustausch, zufällige Ersetzungen, Löschungen und Wortvertauschungen angewendet. Ein erster Testlauf  auf den augmentierten Daten zeigte eine Verbesserung  des Macro Average Recall für die unterrepräsentierten Sublabeln. Allerdings haben die so trainierten Modelle auf den Originaldaten eine deutlich schlechtere Leistung gehabt, weswegen dieser Ansatz verworfen wurde.

\subsubsection{Gram-Schmidt Verfahren zur Erstellung von Embedding}
Das Gram-Schmidt-Verfahren wurde zur Erzeugung von Vektorrepräsentationen eingesetzt, indem $n$ Artikel aus $m$ Kategorien eine Orthonormalbasis für einen $m\cdot n$-dimensionalen Raum definierten. Die Skalarprodukte der verbleibenden Vektoren mit dieser Basis dienten als Koordinaten der Artikel in diesem Raum. Ein sehr ähnlicher Ansatz findet sich bei Yang et al. \cite{Yang2019}. Da die Methode zu einer Verschlechterung der Ergebnisse führte, wurde sie verworfen. Eine detaillierte Beschreibung der Implementierung und der Experimente ist im Skript <Quelle einfügen> zu finden.

\subsubsection{Klassifizierung einzelner Wörter als Kodierung}
Da die ersten vier Ansätze Schwächen in der Multi-Label-Klassifikation zeigten, wurde ein alternativer Ansatz erprobt, bei dem einzelne Wörter klassifiziert und anschließend so kodiert wurden, dass ähnlich klassifizierte Wörter durch nahe beieinanderliegende Vektoren repräsentiert werden. Das Verfahren verlief wie folgt: Zunächst wurden alle Artikel in Wörter zerlegt, die das Label ihres jeweiligen Artikels erhielten. Wörter mit mehreren Zugehörigkeiten wurden als separate Klassen behandelt. Anschließend erfolgte eine Klassifikation mittels eines Modells wie einer Support Vector Machine. Da bereits die Klassifikation einzelner Wörter unzureichende Ergebnisse lieferte, wurde dieser Ansatz nicht weiterverfolgt.


\subsection{Ausbaumöglichkeiten}
Es gibt mehrere Möglichkeiten, um die Ergebnisse dieser Arbeit möglicherweise weiter zu verbessern. Zunächst könnten weitere Vorverarbeitungsmethoden getestet werden. Beispielsweise könnten andere Vektorisierungsmethoden zu einer besseren Repräsentation der Artikel führen. Eine weitere Möglichkeit wären weitere Modellarchitekturen zu testen. Zusätzlich könnte es hilfreich sein, Metadaten wie Autoren, Editoren oder der die Anzahl der Bearbeitungen in die Analyse einzubeziehen, um zusätzliche Kontextinformationen zu nutzen.


\label{Ausblick}


\section{Zusammenfassung und Fazit}
\label{ZusammenfassungUndFazit}
Diese Arbeit untersuchte die automatisierte Klassifikation von Wikipedia-Artikeln hinsichtlich ihres \textit{promotional} (werblichen) Charakters. Dabei wurden zwei Problemstellungen adressiert: die Unterscheidung zwischen werblichen und nicht-werblichen Artikeln sowie die Zuordnung der Labels bei werblichen Wikipedia-Artikeln. Zur Umsetzung wurden fünf Modelle entwickelt und evaluiert. Dabei wurde der originale Datensatz durch Wikipedia-Artikel aus dem Wikipedia-Dump erweitert. Die binäre Klassifikation lässt sich durch logistische Regression, SVM oder neuronale Netze gut lösen. In der Multi-Klassen-Klassifizierung zeigte DistilBERT die besten Resultate. Die Multi-Label-Klassifikation erreichte geringere Werte. Das beste Modell für dieses Problem war der Naive Bayes-Klassifikator mit einem Recall von 66\%. Die Ergebnisse zeigen, dass klassische Methoden des maschinellen Lernens sich für die Erkennung werblicher Wikipedia-Artikel gut eignen. Die Multi-Label-Klassifikation benötigt jedoch noch weitere Verbesserungen, zum Beispiel durch eine bessere Vorverarbeitung oder fortgeschrittenere Methoden.


% References
\addreferences

\makestatement{5}

\end{document}