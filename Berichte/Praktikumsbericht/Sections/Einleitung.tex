\label{Einleitung}
Wikipedia-Artikel werden häufig als Trainingsdaten für verschiedene Modelle des maschinellen Lernens wie z.B. BERT \cite{Devlin2018} und FastText \cite{Bojanowski2016} verwendet. Dabei sind insbesondere sachlich richtige und unvoreingenommene Trainingsdaten erwünscht, die das Training und damit das Modell verbessern.

Das Ziel dieses Projektpraktikums ist die Entwicklung eines Modells, um Wikipedia-Artikel zu klassifizieren. Dabei soll die Qualität der Artikel untersucht und entschieden werden, ob ein Artikel einen \textit{werblichen} Charakter aufweist oder nicht. Dazu wurden verschiedene Modelle des maschinellen Lernens trainiert und evaluiert. Die Vorgehensweise zur Bearbeitung der Problemstellung entsprach dem Data Science Life Cycle. Abschnitt \ref{Aufgabenverteilung} und Abschnitt \ref{Organisation} gehen auf die organisatorischen Aspekte des Projektes ein. Abschnitt \ref{Datensatz} geht auf die Datenanalyse und die Definition der Problemstellung ein. Darüber hinaus wird in diesem Abschnitt auch über weitere hinzugezogene Datensätze gesprochen. In Abschnitt \ref{sec:Ansaetze} werden die verwendeten Modelle vorgestellt.
Abschnitt \ref{Experimente} beschreibt die durchgeführten Experimente und in Abschnitt \ref{Ausblick} werden die Ergebnisse noch einmal zusammengefasst und bewertet.

