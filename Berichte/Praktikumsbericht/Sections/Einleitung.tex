Das Ziel dieses Projektpraktikums ist die praktische Anwendung von Methoden des maschinellen Lernens auf einen vorgegebenen Datensatz aus dem Bereich der Web Science. Wir haben uns für das Thema \textbf{Qualität von Wikipedia-Artikeln} entschieden und nutzen dafür den Datensatz von Kaggle: \url{https://www.kaggle.com/datasets/urbanbricks/wikipedia-promotional-articles}\\
Im Rahmen dieses Projekts bearbeiten wir folgende Teilaufgaben:

\begin{enumerate} 
\item \textbf{Analyse des Datensatzes und Identifizierung einer geeigneten Problemstellung}: Wir untersuchen den bereitgestellten Datensatz eingehend, um ein maschinelles Lernproblem zu formulieren, das mit den vorhandenen Daten gelöst werden kann. 
\item \textbf{Aufbereitung und Vorverarbeitung des Datensatzes}: Wir bereinigen und transformieren die Daten, um sie für die Modellierung vorzubereiten. 
\item \textbf{Anwendung von drei klassischen Methoden des maschinellen Lernens}: Basierend auf den Inhalten der Kapitel 2 und 3 des Kurses \glqq Einführung in Maschinelles Lernen\grqq{} implementieren wir drei klassische Algorithmen, um die identifizierte Problemstellung zu adressieren. 
\item \textbf{Anwendung eines Deep-Learning-Ansatzes}: Wir recherchieren einen geeigneten Deep-Learning-Ansatz, setzen diesen um und wenden diesen auf die Problemstellung an
\item \textbf{Entwickeln eines eigenen Ansatzes}: Im Rahmen dieser Ausarbeitung wird ein eigener Ansatz für die Problemstellung entwickelt und beschrieben. 
\item \textbf{Interpretation und Diskussion der Ergebnisse}: Basierend auf den bisherigen Resultaten entwickeln wir eine neue Idee für einen passenden Ansatz, beispielsweise eine neue Architektur für ein neuronales Netzwerk, die wir implementieren und anwenden. \end{enumerate}