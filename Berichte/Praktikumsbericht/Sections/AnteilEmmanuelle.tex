\label{AnteilEmmanuelle}
Emmanuelle führte bei den regelmässigen Treffen Protokoll. 
%Ausserdem teilte sie den Bericht in verschiedene Teildateien auf, sodass es beim parallelen Arbeiten nicht zu Problemen kam.
Beim Bericht ergänzte sie die Abschnitte \ref{Einleitung}, \ref{AnteilEmmanuelle},\ref{Programmiersprache}, \ref{ToolsUndTechniken} , \ref{UrsprunglicherDatensatz}, \ref{ProblemeDatensatz}, \ref{Problemdefinition}, \ref{Transformer}, \ref{Ausblick}, \ref{ZusammenfassungUndFazit}
%4.3 Weitere Daten
. %Sie schrieb auch beim 5 Abschnitt den einleitenden Text.
Sie ergänzte die Unterabschnitte von 7.1 Erfolglose Versuche.
%Beim Zwischenvortrag entwickelte sie eine erste Version, die als Basis der Struktur des Vortrages dienen sollte. 
Emmanuelle unterstüzte die anfänglichen Analysen und führte weitere durch, um einen geeigneten 5. Ansatz zu finden.
Emmanuelle implementierte den 5. Ansatz, der um Transformer ging. Um dessen finales Konzept zu erarbeiten, machte sie einige Versuche.
%Diese Analysen führten auch zu entsprechenden Versuchen, in der Entwicklung von Ansätzen.
%Außerdem half sie bei der Suche, nach weiteren Datensätzen.
Emmanuelle hielt mit Johannes Krämer den Abschlussvortrag. %Sollen wir die Datensätze noch ergänzen
%Diese Versuche waren allerdings nicht besonders erfolgreich, weil das Ziel der Recherchen entweder nicht erfüllt wurde oder die gefundenen Artikel, die Probleme des Datensatzes nicht gelöst haben. 
%Sie half auch bei der Literaturrecherche für die Datenaugmentation.

%Minimale Unterstützungen der anderen Ansätze, die sie erbrachte, waren das bereitstellen ihrer GPU für das Training des CNN und eine Rechercheunterstützung, um eine Erklärung für die sehr guten Ergebnisse des SVM-Ansatzes zu finden.